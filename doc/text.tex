\documentclass[12pt]{article}
\renewcommand{\baselinestretch}{1.5}

\usepackage{amsmath} % Advanced math typesetting
\usepackage[utf8]{inputenc} % Unicode support (Umlauts etc.)
\usepackage[russian]{babel} % Change hyphenation rules
\usepackage{hyperref} % Add a link to your document
\usepackage{graphicx} % Add pictures to your document
\usepackage{listings} % Source code formatting and highlighting

\begin{document}
\thispagestyle{empty}
\begin{center}
\small{Московский Физико-Технический Институт

Факультет Аэрофизики и Космических Исследований

Кафедра Логистические Системы и Технологии}

\parskip=80pt
КУЗЬМИНА Антонина Ильинична
\parskip=30pt

\textbf{\large{Математическое моделирование конвейера принятия торговых решений трейдером фондовой биржи}}

ДИПЛОМНАЯ РАБОТА
\end{center}

\begin{flushright}
\parskip=50pt
Научный руководитель:
\end{flushright}

\vspace{\fill}
\begin{center}
Москва, 2016
\end{center}
\newpage{}
\tableofcontents{}
\newpage{}
\renewcommand\thesection{}
\renewcommand\thesubsection{}
\renewcommand\thesubsubsection{}
\section{Введение}
\section{Раздел 1. Постановка задач}
\subsection{1.1. Необходимые термины}
\subsection{1.2. Этапы конвейера принятия решений трейдером фондовой биржи}
\subsection{1.3. Постановка задач}
\subsubsection{1.3.1. Задача поиска известного паттерна в истории котировок}
\subsubsection{1.3.2. Задача кластеризации в пространстве фрагментов историй торгов}
\subsubsection{1.3.3. Задача построения эффективной торговой стратегии}
\section{Раздел 2. Теоретическое введение}
\subsection{2.1. Алгоритм динамического искажения времени}
\subsubsection{2.1.1. Базовый алгоритм динамического искажения времени}
\subsubsection{2.1.2. Алгоритм derivative dynamic time warping}
\subsubsection{2.1.3. Метрики расстояния}
\newpage{}
\subsection{2.2. Алгоритмы кластеризации}
\subsubsection{2.2.1. Примеры задач кластеризации}
\subsubsection{2.2.2. Эвристические графовые алгоритмы кластеризации}
\subsubsection{2.2.3. Статистические алгоритмы кластеризации}
\subsubsection{2.2.4. Алгоритмы иерархической кластеризации}
\subsubsection{2.2.5. Самоорганизующиеся карты Кохонена}
\section{Раздел 3. Численные эксперименты}
\section{Заключение}
\section{Список литературы}
\end{document}