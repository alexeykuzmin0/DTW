\documentclass[12pt]{article}
\renewcommand{\baselinestretch}{1.5}

\usepackage{indentfirst}
\usepackage{amsmath} % Advanced math typesetting
\usepackage[utf8]{inputenc} % Unicode support (Umlauts etc.)
\usepackage[russian]{babel} % Change hyphenation rules
\usepackage{hyperref} % Add a link to your document
\usepackage{graphicx} % Add pictures to your document
\usepackage{listings} % Source code formatting and highlighting

\begin{document}
\thispagestyle{empty}
\begin{center}
\small{Федеральное государственное автономное образовательное учреждение высшего профессионального образования \\ «Московский физико-технический институт (государственный университет)»

Факультет Аэрофизики и Космических Исследований

Кафедра Логистические Системы и Технологии}

\parskip=80pt
КУЗЬМИНА Антонина Ильинична
\parskip=30pt

\textbf{\large{Математическое моделирование конвейера принятия торговых решений трейдером фондовой биржи}}

ДИПЛОМНАЯ РАБОТА
\end{center}

\begin{flushright}
\parskip=50pt
Научный руководитель:
\end{flushright}

\vspace{\fill}
\begin{center}
Москва, 2016
\end{center}
\newpage{}
\tableofcontents{}
\newpage{}
\renewcommand\thesection{}
\renewcommand\thesubsection{}
\renewcommand\thesubsubsection{}
\section{Введение}
В последнее время большое количество людей занимается торговлей на бирже — по
некоторым оценкам, около 800 тысяч человек. О популярности этой сферы деятельности
свидетельствует также и объем торгов на московской бирже, растущий день ото дня и
составивший 4.65 трлн рублей 16 декабря 2014 года.

В то же время, игра на бирже является очень рискованным видом деятельности — до 80%
участников торгов терпят убытки. Поэтому трейдер никогда не принимает решений по
наитию, а всегда использует холодный расчет. Вследствие этого, как правило, большинство
торговых стратегий довольно легко формализуемы.

Автоматические торговые системы имеют большое количество преимуществ по сравнению с
трейдером-человеком. Самые значительные из них заключаются в следующем:

\begin{itemize}
\item Торговый робот может торговать круглосуточно, не отвлекаясь на еду и сон.
\item Торговый робот не отнимает ценное время человека — можно запустить
одновременно несколько роботов и, в то же время, заниматься созданием новых
стратегий.
\item Торговый робот принимает все решения в строгом соответствии с логикой алгоритма,
он готов терпеть просадки и не берет на себя лишний риск, что очень важно в
условиях высоковолатильного рынка.
\item Торговый робот способен принимать решения гораздо быстрее любого человека —
если трейдеру для совершения сделки требуется не менее 0.5 секунды, робот способен
совершить сделку уже через 2-3 мкс.
\end{itemize}

Объектом исследования в данной работе является конвейер принятия решений трейдером фондовой биржи.

Основной целью данной работы является разработка математической модели конвейера принятия решений трейдером фондовой биржи, а также ее практическая реализация и проверка на реальных данных как всей моделирующей программы в целом, так и ее отдельных частей.

Для достижения цели настоящей работы поставлены следующие задачи:

\begin{itemize}
\item Исследование существующих методов моделирования работы трейдера.
\item Исследование существующих открытых источников финансовой информации
(котировок различных активов, а также торговых индикаторов) и принятие решения.
об использовании этих данных для построения и тестирования модели конвейера принятия торговых решений трейдером фондовой биржи
\item Исследование существующих открытых источников информации о фигурах технического анализа для использования этих данных при построении модели.
\item Изучение существующей научной литературы по вопросу поиска известных паттернов в истории котировок финансовых инструментов.
\item Выбор метрик расстояния между свечами для использования при реализации алгоритма динамического искажения времени.
\item Реализация алгоритма динамического искажения времени для поиска известного паттерна в истории котировок торгового инструмента в среде программирования Microsoft Visual Studio с использованием языка программирования C\#.
\item Тестирование программы поиска известного паттерна в истории котировок торгового инструмента на реальных исторических данных с использованием методов модульного и функционального тестирования.
\item Определение качества работы программы поиска известного паттерна на реальных исторических данных посредством проведения слепого исследования, а также сравнение достигнутого уровня качества работы при использовании различных метрик расстояния.
\item Изучение существующих методов кластерного анализа.
\item Реализация различных методов кластеризации для поиска новых паттернов в истории котировок финансовых инструментов в среде программирования Microsoft Visual Studio с использованием языка программирования C\#.
\item Тестирование программы поиска новых паттернов в истории котировок торгового инструмента с применением алгоритмов кластеризации на реальных исторических данных с использованием методов модульного и функционального тестирования.
\item Определение качества работы программы поиска новых паттернов в истории котировок торгового инструмента с применением алгоритмов кластеризации на реальных исторических данных посредством проведения слепого исследования, а также сравнение качества работы различных методов кластеризации между собой.
\item Реализация торговой системы, основанной на использовании найденных фигур технического анализа, в среде программирования Microsoft Visual Studio с использованием языка программирования C\#.
\item Тестирование торговой системы, основанной на использовании найденных фигур технического анализа, на реальных исторических данных и определение важнейших параметров этой торговой системы (прибыль, просадка и тд.).
\item Реализация алгоритма-советника для торговой платформы MetaTrader для проверки
торговой системы на демо-счете в режиме реального времени.
\end{itemize}

Теоретической основой исследования явились положения и концепции, представленные в
работах отечественных и зарубежных авторов по проблемам:

\begin{itemize}
\item Численной оптимизации.
\item Поиска паттернов во временных рядах.
\item Алгоритма динамического искажения времени.
\item Классической и вероятностной постановки задач машинного обучения.
\item Эвристических, статистических и иерархических методов кластеризации, в том числе, с использованием нейронных сетей.
\item Переобучения и мультиколлинеарности, а также методам борьбы с этими проблемами:
регуляризации, выделению главных компонент и др.
\item Сравнения качества работы различных алгоритмов машинного обучения.
\item Технического анализа.
\end{itemize}

Вопросы, рассматриваемые в данной работе, нашли отражение в трудах таких классических авторов, как Колмогоров, Вапник, Червоненкис, Закс и Стоун.

Среди современных ученых схожими проблемами занимаются Воронцов, Горбань, Халл, Алексис, Бушерон, Тибширани, Руммельхарт, Носедал, Райт, Куликов.

Работа состоит из введения, трех глав, заключения и списка литературы.

В первой главе приведены экономические термины, необходимые для понимания работы, описаны этапы конвейера принятия решений трейдером фондовой биржи, а также приведена формальная постановка задач, рассматривающихся в работе.

Вторая глава представляет собой теоретическое введение к описанию экспериментов, проведенных в рамках данной работы. В нем описаны алгоритмы, примененные при написании программы: алгоритм динамического искажения времени (а также его модификация, алгоритм derivative dynamic time warping, и используемые метрики расстояния в пространстве историй котировок) и алгоритмы кластеризации (эвристические, статистические и иерархические методы кластеризации, а также самоорганизующиеся карты Кохонена). Приведены основные теоремы, связанные с гарантиями сходимости используемых методов.

В третьей главе описываются результаты, полученные в ходе практической реализации алгоритмов и тестирования их на реальных исторических данных.
\section{Глава 1. Постановка задач}
\subsection{1.1. Необходимые термины}
Ниже приведены определения финансовых терминов, использующиеся в работе.

{\it Актив} --- некоторая сущность, которая может быть куплена или продана в любой момент времени по цене, соответствующей этому моменту времени. Цены покупки и продажи актива в один и тот же момент времени не обязаны совпадать.

{\it Тик} --- сделка купли-продажи, произошедшая на бирже. Характеризуется моментом времени, ценой и объемом. Объем сделки - количество элементарных единиц актива, которые были проданы продавцом и куплены покупателем.

{\it Свеча} --- элемент данных, представляющий собой консолидированную информацию об изменении цены актива в некоторый промежуток времени. Как правило, свеча включает в себя 4 величины: цену открытия интервала (цена первого тика из временного интервала), цену закрытия (цена последнего тика из временного интервала), а также максимальную и минимальную цены тиков из рассматриваемого временного интервала. Нередко также в состав свечи включают общий объем всех сделок, произошедших в течение рассматриваемого промежутка времени, однако в данной работе эта величина не используется.

{\it Торговая система} --- алгоритм, совершающий сделки на бирже по определенным математическим правилам. Может иметь параметры, влияющие на поведение системы.

{\it Сделка покупки актива} --- сделка по покупке-продаже актива, в которой рассматриваемая торговая система выступает в качестве покупателя.

{\it Сделка продажи актива} --- сделка по покупке-продаже актива, в которой рассматриваемая торговая система выступает в качестве продавца.

{\it Закрытая сделка} --- пара сделок с совпадающими объемами, состоящая из сделки по покупке актива и сделки по продаже актива.

{\it Ряд данных} --- последовательность данных о цене актива за определенный промежуток времени. Как правило, включает в себя информацию обо всех свечах этого актива за данный промежуток времени.

{\it Прибыль закрытой сделки} --- разность цен сделок продажи и покупки этой закрытой сделки, умноженная на объем этих сделок.

{\it Прибыль торговой системы за некоторый интервал времени.} Обозначим $n$ общее число закрытых сделок торговой системы за рассматриваемый период. Обозначим $p_i,~i=1,...,n$ прибыль $i-$ой закрытой сделки. Тогда прибылью торговой системы называется величина $profit=\sum\limits_{i=1}^n{p_i}$.

{\it Просадка торговой системы за некоторый интервал времени.} Обозначим $n$ общее число закрытых сделок торговой системы за рассматриваемый период. Обозначим $p_i,~i=1,...,n$ прибыль $i-$ой закрытой сделки. Тогда просадкой торговой системы называется величина $$drawdown=\max\limits_{i=1,...,n}\left(\max\limits_{k=1,...,i}\sum\limits_{j=1}^kp_j - \sum\limits_{j=1}^ip_j\right)$$

{\it Функционал качества торговой системы} --- некоторая функция, характеризующая качество торговой системы. Как правило, для ее вычисления используется последовательность закрытых сделок торговой системы. Типичные примеры функционала качества --- $profit$ и $profit/drawdown$.
\subsection{1.2. Этапы конвейера принятия решений трейдером фондовой биржи}
В данной работе рассматривается трейдер фондовой биржи, принимающий торговые решения на основе фигур технического анализа. Конвейер принятия решений в этом случае включает в себя следующие этапы:

\begin{enumerate}
\item Поиск закономерностей фондового рынка
\begin{enumerate}
\item Выделение типичных фигур технического анализа
\item Определение информативности каждой фигуры технического анализа
\end{enumerate}
\item Создание торговой стратегии
\begin{enumerate}
\item Поиск фигур технического анализа в биржевых данных в режиме реального времени
\item Принятие торгового решения и совершение сделки
\item Оптимизация торговой стратегии
\item Запуск автоматической торговой системы
\end{enumerate}
\end{enumerate}

В данной работе рассматриваются все этапы этого конвейера, однако основное внимание уделяется трем задачам: задаче поиска известного паттерна в истории котировок, задаче кластеризации в пространстве фрагментов историй котировок и задаче автоматизированного построения эффективной торговой стратегии.
\subsection{1.3. Постановка задач}
\subsubsection{1.3.1. Задача поиска известного паттерна в истории котировок}
\subsubsection{1.3.2. Задача кластеризации в пространстве фрагментов историй торгов}
\subsubsection{1.3.3. Задача построения эффективной торговой стратегии}
\section{Глава 2. Теоретическое введение}
\subsection{2.1. Алгоритм динамического искажения времени}
\subsubsection{2.1.1. Базовый алгоритм динамического искажения времени}
\subsubsection{2.1.2. Алгоритм derivative dynamic time warping}
\subsubsection{2.1.3. Метрики расстояния}
\newpage{}
\subsection{2.2. Алгоритмы кластеризации}
\subsubsection{2.2.1. Примеры задач кластеризации}
\subsubsection{2.2.2. Эвристические графовые алгоритмы кластеризации}
\subsubsection{2.2.3. Статистические алгоритмы кластеризации}
\subsubsection{2.2.4. Алгоритмы иерархической кластеризации}
\subsubsection{2.2.5. Самоорганизующиеся карты Кохонена}
\section{Глава 3. Численные эксперименты}
\subsection{3.1. Поиск паттерна в истории котировок}
\subsection{3.2. Кластеризация фрагментов историй котировок}
\subsection{3.3. Построение полностью автоматизированной торговой стратегии}
\section{Заключение}
\section{Список литературы}
\end{document}